\documentclass{beamer}
\usepackage{graphicx}
\title{CS 562 Final Project}
\author{Olivia Eng, Jakob Gibson and Nick Cali}
\date{}

\usetheme{CambridgeUS}
\begin{document}
    \begin{frame}
        \begin{titlepage}
            
        \end{titlepage}
    \end{frame}

    \begin{frame}
        \frametitle[Overview]{Problem Overview}
        \begin{itemize}
            \item SQL queries are powerful tools
            \item basic SQL syntax makes expressing complicated or interconnected OLAP queries hard
            \item this is because one cannot "decouple" the formation of groups (the GROUP BY clause) and the computation of aggregates (such as MAX, MIN, COUNT, SUM, and AVG) 
            \item \begin{itemize}
                \item i.e. one must, in standard SQL, compute these things in distinct queries and join them all together at the end
            \end{itemize}
        \end{itemize}
    \end{frame}

    \begin{frame}
        \frametitle[Solution]{Solution}
        Luckily, there is a solution! Introduced in two papers:
        \begin{enumerate}
            \item \textit{Querying Multiple Features of Groups in Relational Databases} by D. Chatziantoniou and K. Ross
            \item \textit{Evalutaion of Ad Hoc OLAP: In-Place Computation} by D. Chatziantoniou
        \end{enumerate} 
        the $\varPhi$ operator (an extension to relational algebra) is a way for us to express ourselves (and these complex OLAP queries) concisely.\par
        However, this being an addition to relational algebra, does not have a direct implementation in SQL (that is, it is not found in any standard implementation of SQL [MySQL, PostgreSQL, etc.])
    \end{frame}

    \begin{frame}
        \frametitle[Demo Points]{Moving Forward: Points to Be Covered in Demo}
    \end{frame}

\end{document}